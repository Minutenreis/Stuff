\documentclass[sigconf]{acmart}
\usepackage{booktabs}

\begin{document}

%todo: look into title
\title{Parallelising Hybrid MPI Programms efficiently}
\author{Dreßler Justus}
\orcid{0009-0005-0657-5854}
\affiliation{
    \institution{Friedrich Schiller University}
    \department{Faculty of Mathematics and Computer Science}
    \streetaddress{Fürstengraben 1, Main University Building (Universitätshauptgebäude)}
    \postcode{07743}
    \city{Jena}
    \country{Germany}
}
\email{justus.dressler@uni-jena.de}

\keywords{MPI, MPI+OpenMP,MPI+Threads}

\begin{CCSXML}
    <ccs2012>
    <concept>
    <concept_id>10010147.10010169</concept_id>
    <concept_desc>Computing methodologies~Parallel computing methodologies</concept_desc>
    <concept_significance>500</concept_significance>
    </concept>
    </ccs2012>
\end{CCSXML}

\ccsdesc[500]{Computing methodologies~Parallel computing methodologies}

\begin{abstract}

    %TODO: add abstract
    Lorem Ipsum dolor sit amet, consectetur adipiscing elit.
    Mauris dictum quam eget orci porttitor ultrices.
    Suspendisse purus nisl, dapibus et rhoncus in, eleifend id quam. Integer volutpat dolor urna, non auctor magna sodales non.
    Ut in fermentum erat, lacinia malesuada urna.
    Nullam et dolor ut purus fermentum rutrum.
    Suspendisse non velit sollicitudin, rutrum est at, commodo sem. Pellentesque viverra tincidunt mauris ut egestas.
    Integer vel orci sapien. Sed ut mi at lacus vehicula fermentum.
    Integer euismod vestibulum auctor.
    Nunc vulputate leo eget aliquam bibendum.

\end{abstract}

\maketitle

% Introduction, roughly 1 page
% todo: rework
\section{Introduction}

% Historisch MPI-Everywhere
% heute MPI+OpenMP (warum)
% Was sind die schwierigkeiten mit MPI+OpenMP
% Paper versucht Lösungen zusammenzufassen
Message Passing Interface is the de facto standard for scaling High Performance Computing (HPC) Workloads across nodes.
Historically applications relied on the memory wasteful model of MPI-Everywhere to achieve good scalability across nodes.
In the last decades however a rapid increase in core counts per CPU neccessitated a switch to MPI+OpenMP to better use newer CPU's ressources.
While this did work, it brought fourth new challenges in designing software for HPC workloads.
The main pain point is the much harder to achieve parallel communication in MPI+OpenMP compared to MPI-Everywhere.
The newest MPI version (4.0) introduced new ways to expose communication parallelism in MPI software to ease this pain point.
So this paper tries to give an overview over the current ways to expose parallel communication to build scalable MPI+OpenMP software.
This includes a brief theoretical look at each method as well as sample code so developers may apply these methods to their own software.

% this feels like making a buzzword list

% Problemstellung: Wie expose ich logisch Parallele Nachrichten an eine MPI Biblithek (MPICH, OpenMPI) sodass diese es auf parallele Kommunikationswege mappen kann?


\section{Motivation}
The historically preferred method of MPI-Everywhere made logical parallelism easy to express in software.
%todo: cite or rework
It did however have massive problems with memory management.
Classical MPI-Everywhere programs spawn a single process per core available.
MPI+OpenMP programs normally spawn a single process per node or NUMA domain \cite{zambreLessonsLearned2022}.

%todo: cite MPI 3 Standard
A MPI communicator is a construct that defines which processes may talk with each other.
Each MPI process is required to keep a list of all participating MPI processes in the default communicator \verb|MPI_COMM_WORLD|.
Assuming an $n$ core processor $n$ copies of the list are lying in each nodes memory in the MPI-Everywhere model.
An equivalent MPI+OpenMP program shares its list between its cores and also spawns $n$ times less processes, so \verb|MPI_COMM_WORLD| is $n$ times smaller and $n$ times less often on each processor.
For current generation processors exceeding 100 cores like Intel® Xeon® 6 \cite{intelXeon6} or 4th Generation AMD EPYC™ \cite{amd4thGenEpyc} this equates to a reduction of \verb|MPI_COMM_WORLD|'s memory consumption of over $n^2 > 10.000$ times.

Each MPI process also needs to be a full program, which means each core has to have all instructions duplicated.
OpenMP however only needs to generate code for the actually parallised parts (typically loops) of the program, which reduces the size for all additional cores significantly.

MPI+OpenMP programs do, however, tend to perform slower than MPI-Everywhere versions in practice \cite{zambreLessonsLearned2022}.
This is caused in part by accidental synchronization while calling the MPI library.
The MPI standard mandates a high amount of serialization to ensure correct communication, for example all messages on the same \verb|<Communicator,Rank>| pair are transmitted in non-overtaking order \cite{mpi40}.
This is important to prevent deadlocks (two processes each wait on the other) but also slows down the communication overall.
We will explore how to circumvent such slowdowns in the following section.


% Body, roughly 2.5 pages
\section{Parallel Point-to-Point Communication in MPI 4.0}

Point-to-Point Communication refers to any communication between 2 different processes.
There are several ways to implement logical parallel point-to-point communication efficiently in MPI.
For MPI versions lower than 4.0 a unique communicators per logical parallel communication channel may be used.
MPI 4.0 introduced Partitioned Operations to aid efficient parallel communication.
In Partitioned Operations multiple threads each contribute part of a larger message.
MPI 4.0 also added new hints, a way for a program to guarantee MPI that it will not use certain features, that can make tags viable for performant parallel communication.
% todo: MPI Sessions yes or no? -> Read up if time, don't forget graphics and code before though
Furthermore MPI 4.0 introduced MPI Sessions, which may be used for parallelisation purposes, but are not considered in the scope of this paper.
In the following subsections this paper will discuss the mentioned methods in more detail.

\subsection{Communicator}

%todo: is simple the right word? most backwards compatible?
The simplest though maybe not most intuitive way to parallelise communication is to assign each logical parallel message its own communicator.
Since MPI gives no implies no relative ordering on different communicators \cite{zambreLessonsLearned2022}, libraries can map messages on different communicators to different underlying message channels.
To use different communicator an MPI application developer needs to be aware of a few key problems:

1) For optimal parallelisation there needs to be two communicator for each pair of threads communicating between different processes.
%todo: insert figure, also link figure
%todo: possibly code for that same figure so the complexity can be better judged by a reader?
To demonstrate the need for two communicator per pair of threads imagine a problem devided into a line of Nodes.
If only one communicator is used per pair of threads, communication between \verb|<Rank_0,Thread_5>| and \verb|<Rank_1,Thread_3>| shares its communicator with the communication between \verb|<Rank_1,Thread_5>| and \verb|<Rank_2,Thread_3>|.
This is problematic since the messages from \verb|Rank_0| and \verb|Rank_2| to \verb|Rank_1| now run on the same communicator and with that those two exchanges are no longer independent.

2) For large multidimensional stencils the number of communicators needed grows rapidly and an MPI program using those should consider other methods to expose parallelisation.
For more complex stencils like a 27-point 3D stencil as seen in Hypre\cite{hypre2020} on a larger 64-core processor (data distributed as [4,4,4] chunks) the number of communicators needed to expose all possible communication parallelism amounts to $6 \cdot 4^2$ for sides plus $24 \cdot 4^2 - 8$ for the corner diagonal plus $24 \cdot 4^2 - 48$ for the edge diagonals, totalling $808$ communicators.
The workload would only need a single parallel communication channel for each thread communicating between nodes ideally though, which is only $4^3 - (4-2)^3 = 56$.
The equations were derived from those used by Zambre and Chandramowlishwaran \cite{zambreLessonsLearned2022}.
Generating so many communicators is not only complex, it also strains MPI libraries that have to map the many possibly parallel messages to a far smaller number of parallel paths in hardware (as an example Omni-Path only has 160 hardware contexts \cite{intelOmniPath}).
Since the MPI library lacks the knowledge of the intended communication pattern its mapping to the relevant hardware contexts will be suboptimal.

3) Using communicators to expose parallelism and group processes introduces performance loss or limits the portability across MPI libraries.
Since MPI 4.0 lacks standardised hints regarding communicator usage, MPI libraries cannot differentiate between communicators used to expose parallelism and those that are used for grouping processes together.
If a MPI library tries to parallelise across communicators, any communicator that is not used to expose parallelisation puts unneccessary strain on the library, reducing overall performance.
%todo: find example hint in MPICH or OpenMP
A program may use library specific hints to specify which communicators are used for what purpose, but doing so makes the program dependent on that MPI library instead of being compatible with any MPI library, at least for the optimal performance \cite{zambreLessonsLearned2022}.

4) Using communicators for irregular access patterns degrades performance.
Imagine a taskbased application with $n$ worker threads and one main thread in each process.
%todo: generate new graphic like "Lessons Learned on MPI+Threads" Fig. 5 but with N nodes to better demonstrate the infeasability of "Duplicating communicators per communication partner"
All workers in one process may need to communicate with the main thread of any other process.
In such a case the software may either generate $n$ communicator for each worker thread in a process or generate $n \cdot m$ communicator for all $n$ worker in a process and all $m$ processes.
The latter version is infeasable for any non trivial number of processes as it strains the MPI library for the same reason large multidimensional stencils do.
So the software is realistically restricted to just $n$ communicators.
This then forces the main thread to communicate over the same communicators as its workers.
And as the main thread needs to iterate over all possible communicators it will contend with its workers to access the communicators and slow down the application that way.
This leads to communicators performing worse than other parallelisation alternatives.
As example communicators performed 1.63x worse than tags with hints in the task-based framework Legion \cite{zambreLogicalParallel2021}.

\subsection{Tags with Hints}

MPI operations like \verb|MPI_Send| take a \verb|<Rank,Tag,Communicator>| triple as input.
The tag is sent to differentiate messages between the same processes on a single communicator.
Normally the MPI standard requires a non-overtaking order for all messages with the same \verb|<Rank,Communicator>| \cite{mpi40}.
So the default settings do not enable a program to use the additionally sent tags for parallelisation.
If the program however guarantees MPI, that it will not use the \verb|MPI_ANY_TAG| (a parameter matching all tags), MPI libraries can send different messages between the same ranks but differing tags in arbitrary order.

To use this new avenue to expose parallelism, programs need to use the \verb|mpi_assert_no_any_tag| hint.
Now any MPI library is free to parallelise over different tags.
To use that a program can simply encode the source thread and target thread into the tag.
A possible implementation is given in FIG. %todo: add figure of sample code
Using tags comes with some caveats:

As with using communicators to expose parallelism, tags are also used in other contexts than parallelisation.
So as with communicators a program has to rely on MPI library specific extra hints to clarify which parts of the tag encodes the parallelisation information or risk a suboptimal distribution to the hardware communication channels.
An example of such extra hints is shown in FIG for the MPI library MPICH. %todo: add MPICH figure

Tags are restricted to 64 bits.
Depending on the programs other uses for tags this may be a non issue or a dealbreaker.
Encoding source and target thread in tags for $n$ core processors takes $2\log_2{n}$ bits.
For a 128 core processors this would mean, that a program reserves $2\log_2{128} = 14$ bits of their tags.
For most programs this will be irrelevant, but if the program already encodes much information in their tags, exposing parallelisation information might not be feasable.

\subsection{Partitioned Point-to-Point Communication}

%todo: add sample code

Partitioned Operations are point-to-point operations where the message is partioned into multiple parts.

To implement partitioned point-to-point operations is slightly more difficult than simple \verb|MPI_Send| and \verb|MPI_Recv| calls.
In contrast to those all operations have to be first initialised on sender and receiver side with \verb|MPI_PSend_Init| and \verb|MPI_PRecv_Init| respectively.
The initialisation may be done long before the operation is executed.
To start the execution there must be a single call to \verb|MPI_Start|.
Then each contributing sending thread may write its part of the message in the shared buffer and then mark its part as ready by calling \verb|MPI_Pready|.
On the receiving side threads may poll for completion of individual partitions of the message by calling \verb|MPI_Parrived|.
As soon as those return a truthy flag a thread may work on that partitions data.
There must be a single final call to \verb|MPI_Wait| or \verb|MPI_Test| to complete the operation, before it may be restarted with another call to \verb|MPI_Start|.

This a bit more complicated setup allows an MPI library to transmit the partitions in parallel instead of waiting for all threads to finish working.
Partitioned operations are ideally suited to very regular communication, as found in stencil applications.
They are very ill suited for irregular communication though, as the operations have to be initialised before any communication may happen \cite{zambreLessonsLearned2022}.
Irregular communication is further complicated by the missing ability to provide wildcards (\verb|MPI_ANY_TAG|, \verb|MPI_ANY_SOURCE|) to \verb|MPI_PRecv_Init| \cite{mpi40}, on which modern task-based runtimes like Legion rely \cite{zambreLessonsLearned2022}.



% see Presentation
% \subsection{Sessions}

%vll. Wort zur Umsetzung in Bibliotheken? OpenMPI und MPICH haben es beide immerhin ...

% End, roughly 0.5 pages
\section{Conclusions}
\bibliographystyle{ACM-Reference-Format}
\bibliography{citations}

\end{document}
